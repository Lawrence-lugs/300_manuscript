\chapter{Introduction}

The use of artificial intelligence (AI) in extreme edge devices such as wireless sensor nodes (WSNs) will greatly benefit the scalability and application space of such nodes. AI can be applied to solve problems with clustering, data routing, and most importantly it can be used to reduce the volume of data transmission via data compression or making conclusions from data within the node itself \cite{alsheikh2014machine}.

However, since devices in the extreme edge are constrained to work with extremely low amounts of memory and energy \cite{Ma_2019}, even the simplest AI models are difficult to execute with typical sequential processors. WSNs have memories in the order of kB and clock speeds in the order of kHz to MHz due to energy constraints, rendering them unable to run state-of-the art AI applications.

A promising hardware approach allowing the use of AI in low-power edge devices is in-memory computing (IMC) \cite{Patterson_1997}. IMC allows very high energy savings compared to other approaches by bypassing the most energy-expensive and time-consuming part of AI processing: memory accesses. Analog IMC with memristors has proven to be fast and efficient at multiply-and-accumulate operations (MACs) which are by far the most common operation used by AI software. Additionally, since memristors are nonvolatile memory devices, they are particularly robust to energy interruptions from ultra-low power situations in WSN. 

In terms of application, work on Analog IMC architectures are usually demonstrated with only very specific and small architectures \cite{Xue_2019,Liu_2020,Mochida_2018}. One very recent one allows a wide variety of existing AI architectures, such as existing computer vision networks, voice recognition, and image recovery, but is way larger and inefficient for it. These networks are made to accomplish more difficult AI applications than what may be desired to be used in the extreme edge, in contrast. For example, ResNets are powerful enough to achieve great accuracy (\>70\%) on the 4 million image dataset ImageNet, while we may only want to use AI for low-resolution images, or temporal sensor data \cite{He_2016}. 

Works on IMC architectures usually focus on improving upon one another by solving the more hardware-related problems and increasing the FLOPs/W and FLOPs/s as much as possible. These metrics assume full-utilization of the Analog IMC computing cores, rendering them slightly inaccurate when the model consumes utilizes a relatively small fraction of the available resources, because some major contributors to the consumption like the driving energy cost the thick bitline wires stay constant.

Left untouched, Analog IMC may remain impractical to use for extreme edge devices, as existing designs are too inflexible or expensive. 

It is thus desirable to co-design an IMC architecture with a weaker, smaller class of AI models. It must be flexible and large enough to accomodate the entire class, but small enough to have a high utilization. High utilization further saves area and static energy on the edge device if we were to co-design the analog IMC architecture with said class of AI models. 